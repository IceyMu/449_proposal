Diffusion MRI provides a non-invasive tool for probing the microstructure of spinal cords. Understanding the correlations between spinal cord histology and MRI provides experimental validation of proposed biomarkers as well as a clearer understanding of when MRI is practical for use in diagnosing and predicting the recovery of spinal cord injuries. %
%
Image reconstruction of the microstructure from the MRI data requires a prior mathematical model of the structure to fit to. The accuracy and sensitivity of these models in the presence of extremely damaged tissue such as the kind found in severe spinal cord injury has yet to be validated. %
%
\if{false}
The accuracy and sensitivity of these parameters to the actual histology has yet to be fully explored.
\fi
%
We propose a study on 10 spinal cords donated by patients who suffered fatal injuries to the International Spinal Cord Injury Biobank we will map dMRI parameters in regions of interest and find correlations in the regions between healthy and damaged spinal cord tissue.