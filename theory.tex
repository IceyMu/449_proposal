% NMR / MRI Theory
% DMRI Theory and measurement
% Using Biophysical Models to make sense of MRI data through essentially solving for fit parameters

% ^^ This will be more than enough content

%\subsection{Nuclear Magnetic Resonance}

\subsection{Magnetic Resonance Imaging}
\mri is the use of nuclear magnetic resonance to probe the structure of a sample. Protons neutrons are particles with a non-zero spin angular momentum and therefore ground state nuclei containing an odd number of protons have a net magnetic moment and in the presence of a magnetic field there is a lifting of the degeneracy of the spin-up and spin-down magnetic substates causing it to be energetically preferable for the spins of the particles to be oppositely aligned, this is known as the Zeeman effect, the energy splitting is given by equation \ref{eq:zeeman}.
\begin{equation}
    \Delta E = \gamma \mu B_0
    \label{eq:zeeman}
\end{equation}

Where $B_0$ is the strength of the applied magnetic field, $\mu$ is the nuclear magneton, and $\gamma$ is the gyrometric factor which depends on the angular momentum of the particle. \cite{griffiths2005introduction} In practice, nuclei feel the external field and internal fields from chemical bonding causing a shift in $\Delta E$ and these shifts are catalogued into functional groups. If we consider a proton rich sample, individual protons can emit and absorb photons that match the energy splitting to transition between states, the energy splitting can be determined from measuring the absorption spectrum of the sample where the most energy will be absorbed at the frequency that matches $\Delta E$, this effect is known as nuclear magnetic resonance. \cite{bushberg2011essential} The total energy absorbed is proportional to the number of protons that have been excited therefore when $\Delta E$ is known, a non uniform $\vec{B}$ field can be applied such that the sample will only resonate with the sample in a single plane, choosing a plane is known slice selection, measuring the absorbed energy we learn the proton density in that slice.

Another observable signal is the photons released by the protons as the sample lose magnetization and relaxes back to thermal equilibrium. The time change in magnetization, $\vm$, is given by the equation \ref{eq:bloch} the Bloch equation where we assume $\vec{B_0}$ is pointing along the z-axis:

\begin{equation}
    \frac{d \vm}{dt} = \gamma \vm \times \vec{B_0} - \frac{M_x \hat{x} + M_y \hat{y}}{T_2^*} - \frac{M_z - M_0}{T_1}\hat{z}
    \label{eq:bloch}
\end{equation}

Where $M_0$ is the z-component at thermal equilibrium, $T_1$ is the spin-lattice relaxation constant, and $T_2^*$ is the spin-spin relaxation constant and depend on the structure of the tissue. \cite{bushberg2011essential} Sequences of radio frequency pulses can be used to flip the magnetic moments of the protons and then a gradient can then be applied within the slice to cause the magnetic moments to precess at specific frequencies and phases, this information is encoded in the radio signals produced during relaxation and the measured \mr signals form a k-space matrix of which the Fourier transform yields the real space image of the sample.


% The photon energy has to match the energy splitting

% This preferential absorption spectrum is known as nuclear magnetic resonance

% From NMR, and prior knowledge of the absorption frequency we choose a non uniform B field such that resonance condition is only met in a small region usually a plane. The energy absorbed can be measured with a pickup coil

\subsection{Diffusion MRI}
% Maybe say something about the fundamental limits of conventional MRI
% I have to define all or most of the parameters that the following models depend on

Equation \ref{eq:bloch} assumed stationary protons. If we consider the fact that the protons have the ability to diffuse then we modify equation \ref{eq:bloch} to get equation \ref{eq:torrey} the Bloch-Torrey equation:

\begin{equation}
    \frac{d \vm}{dt} = \frac{d \vm_{\text{Bloch}}}{dt} + \nabla \cdot D \nabla \vm
    \label{eq:torrey}
\end{equation}

Where ${d \vm_{\text{Bloch}}}/{dt}$ is the RHS of equation \ref{eq:bloch} and $D$ is the diffusion tensor. \cite{rohmer2006bloch}

% We will define and assume gaussian diffusion for this section.

The solution to this equation can be simplified as:

\begin{equation}
    \vm = \vm_{\text{Bloch}} e^{-\gamma^2 \delta^2_d \Delta \vec{G}^T D \vec{G}}
    \label{eq:bt_sol}
\end{equation}

Where $\Delta$ is the time between the beginning of each gradient pulse, $\delta_d$ is the length of the pulse, and $\vec{G}$ is the gradient applied to the slice. The factors in the exponent of equation \ref{eq:bt_sol} can be combined into a single factor, $b$, and used as a measure of the sequence's sensitivity to diffusion in the sample by defining:

\begin{equation}
    b = \gamma^2 \int_0^{TE} \int_0^t G(t') dt' dt
\end{equation}

Where $TE$ is the echo time which is the time between the exciting pulse and the measured signal. When time between pulses in an MRI sequence is long enough for diffusion to have a significant effect on the magnetization of the sample the dephasing pulse will no longer be rephased because diffusion will have changed the positions of the protons and therefore the phase difference in the slice will be proportional to the distance the protons have travelled. The effect of diffusion on the phase of the magnetic moments are illustrated in figure \ref{fig:phase} and the quantities $TE$, $\vec{G}$, $\delta_d$, and $\Delta$ are visualized in for a \dmri sequence in figure \ref{fig:sequence}.

\begin{figure}
    \includegraphics[width=0.5\textwidth]{figures/phase.png}
    \caption{Spin phase of the magnetic moment of nuclei before and after the phase is flipped by a radio sequence pulse. Phase is represented by the arrows and all phases are aligned before the radio pulse, nuclei are coloured based on their initial position. (A) shows the effect diffusion has on the overall phase of the sample compared to (B) which would be the case for completely stationary nuclei. Image adapted from \cite{mori2007introduction}}
    \label{fig:phase}
\end{figure}

\begin{figure}
    \includegraphics[width=0.5\textwidth]{figures/sequence.png}
    \caption{Visualization of a spin-echo diffusion weighted Stejskal-Tanner sequence. Top row gives the timing and phase of the \dmri phasing pulses, center row gives the timing amplitudes and duration of magnetic gradient fields, bottom row approximates the signal from resonating tissue, $\Delta$ can be read as the time between the starts of the gradient pulses, $\delta_d$ can be read as the duration of each gradient pulse, $TE$ is the time between the excitation and collection of an MR signal. Image adapted from \cite{rohmer2006bloch}}.
    \label{fig:sequence}
\end{figure}

\subsection{Biophysical Models of Spinal Cord Microstructure}
Determining the structure of the spinal cord from \dmri data requires fitting the data to a prior model of the microstructure. The ones used in our analysis are:

\subsubsection{Diffusion Tensor Imaging}
\dti is a model where the signal in each voxel is fit to gaussian diffusion parameterized by anisotropic gaussian diffusion tensors fitting the form of equation \ref{eq:dti}.

\if{false}
\dti is a model where isortropic gaussian diffusion coefficients are replaced by anisotropic diffusion tensors by the measured signal is assumed to be given by equation \ref{eq:dti}
\fi

\begin{equation}
    S(\hat{g_i}) = S_0 e^{-b \hat{g}_i^T D \hat{g}_i}
    \label{eq:dti}
\end{equation}

Where $D$ is the diffusion tensor, $\hat{g_i}$ is the normalized diffusion gradient, $S_0$ is the expected signal from unrestricted gaussian diffusion. Here the diffusion tensor is a symmetric matrix and therefore requires measuring at least 6 gradient directions to solve for each element. \cite{vedantam2014diffusion}

\subsubsection{Diffusion Kurtosis Imaging}
\dki models non-gaussian diffusion through a Taylor expansion of the signal with respect to the b-value:

\begin{equation}
    \ln[S(b)] = \ln[S(0)] - b D + \frac{b^2}{6} D^2
\end{equation}

Where now $D$ is the kurtosis vector and is represented as a fully symmetric rank 4 tensor which gives 15 independent components. \cite{jensen2005diffusional}

\subsubsection{Spherical Mean Technique}
\smt modelling measures the underlying tissue using data at a fixed b-value using a parameter that is invariant of axon orientation and therefore does not have to make simplifying assumptions about the crossing or dispersion of the fibre anatomy. First we define a normalized diffusion signal:

\begin{equation}
    e_b(\vec{g}) = \frac{S_b(\vec{g})}{S_0}
\end{equation}

We then consider $\bar{e_b}$ which is defined as the spherical mean of $e_b$ averaged over every gradient direction.

\begin{equation}
    \bar{e_b} = \frac{1}{4\pi} \int_{S^2} e_b(\vec{g}) d\vec{g}
\end{equation}

Where the integral is over the surface of a unit sphere. $\bar{e_b}$ is then related to the diffusion coefficient parallel and perpendicular to the axons through:

\begin{equation}
    \bar{e_b} = \exp(-b \lambda_\perp) \cdot
    \frac{\sqrt{\pi}\erf \left(\sqrt{b (\lambda_\parallel - \lambda_\perp)}\right)}
    {2 \sqrt{b (\lambda_\parallel - \lambda_\perp)}}
\end{equation}

Where $0 \leq \lambda_\perp \leq \lambda_\parallel \leq \lambda_\text{free}$, where $\lambda_\perp$ and $\lambda_\parallel$ are the diffusion coefficients perpendicular and parallel to the axons and $\lambda_\text{free}$ is the gaussian diffusion coefficient in the absence of any axons. \cite{kaden2016quantitative}

\subsubsection{Neurite Orientation Dispersion and Density Imaging}
Neurite Orientation Dispersion and Density Imaging (NODDI) is a model that separates the tissue into three compartments: the intracellular region, the extracellular region, and the \csf. The measured signal is modelled as:

\begin{equation}
    S = (1 - \nu_{iso})(\nu_ic S_{ic} + (1- \nu_{ic}) S_{ec}) + \nu_{iso} S_{iso}
\end{equation}

Where $S_{ic}$ and $S_{ec}$ are the diffusion signals due to the intra and extra cellular regions, $S_{iso}$ is the isotropic diffusion in the \csf compartment, and $\nu_{ic}$, $\nu_{ec}$, and $\nu_{iso}$ are their respective volume fractions. The intracellular compartment is modelled as a set of cylinders with zero radius such that diffusion can only occur along its length and is given by:

\begin{equation}
    S_{ic} = \int_{S^2} f(\vec{n}) e^{-b d_\parallel (\vec{g} \cdot \vec{n})^2} d\vec{n}
\end{equation}

Where $f(\vec{n})$ is the orientation density for axons being aligned with the direction of $\vec{n}$ and $d_\parallel$ is the diffusivity of water along the length of the axon. The extracellular component of the signal is modelled as anisotropic gaussian diffusion like that discussed in the \dti section, and $S_{iso}$ is modelled as isotropic diffusion discussed in the \dmri section. \cite{zhang2012noddi}

\subsubsection{ActiveAx}
ActiveAx

\subsubsection{Diffusion Basis Spectrum Imaging}
\dbsi is a model where we consider the \dmri signal to be a linear combination of anisotropic compartments and a spectrum of isotropic diffusion tensors where the signal is given by:

\begin{IEEEeqnarray*}{rCl}
    S(\hat{g}_i) &=& \sum_{n=1}^{N_{\text{Aniso}}} f_n e^{-b_i} \lambda_{\perp, n} e^{-b_i (\lambda_{\parallel, n} - \lambda_{\perp, n}) \cos^2(\psi_{i, n})}\\
    && + \int_a^b f(D) e^{-b_i D} dD
\end{IEEEeqnarray*}

Where we need to fit $N_\text{Aniso}$ the number of anisotropic tensors, $\lambda_{\perp, n}$ and $\lambda_{\parallel, n}$ the diffusivities of the nth anisotropic tensor, $a$ and $b$ the diffusivity limits for the diffusion spectrum, $D$, and $f_n$ is the signal intensity fraction for the nth tensor. \cite{wang2011quantification}














\if{false}
Charged particles such as protons that posses a non-zero spin have an intrinsic angular momentum and this gives it a magnetic moment. The presence of a magnetic field lifts the energy degeneracy of the orientation of the magnetic substates of having the spin aligned or anti aligned with the direction of the magnetic field, this is known as the Zeeman effect. Such particles in a magnetic field can absorb or release electro magnetic radiation in order to transition between the two states, such photons have an energy given by the energy separation between the two states.

% Nuclear magnetic resonance.
Consider a large collection of protons in equilibrium with at temperature T, the ratio of spin up to spin down is given by the boltzmann factor and is small even when multiplied by avagadros number, if we choose a magnetic field on the order of a few tesla the frequency involved to get an appreciable energy is on the order of about 100MHz which is radio frequency we need magnetic fields on the order of a few tesla to get to an appreciable energy absorbed.

We can count the number of flipped protons through energy loss in a pickup coil. Scanning a range of radio frequencies and then detecting the resonance where we get the strongest is how one can determine the magnetic moment of the sample.

It turns out that the lamor frequency of atoms is dependent on the chemical environment as protons will feel the applied external field and the internal fields from chemical binding. This change in the lamor frequency gives a shift in the resonant frequencies which can be cataloged into the functional groups. When these are known NMR can be used to identify the functional groups in an unknown sample.

This idea can be inverted if one knows the chemical makeup of the sample a non uniform B field can be applied to the sample such that protons only experience resonance with the provided radio waves at known locations this technique is known as MRI and allows for the mapping of the proton density in various tissues.

\subsubsection{Diffusion MRI}

Unfortunately the MRI technique described above cannot image tissue at the resolutions necessary to directly map out the microscopic structure of cells and intracellular space. We can still probe the structure tissues at these small scales by sensitizing our MRI scheme to be sensitive to the motion of water molecules throughout the tissue.

This relies on the fact that when protons excited by a radio pulse take a finite amount of time to relax, now the actual vector magnetization of the entire sample is given by the Bloch equations which has a precession term for the around the orientation of the magnetic field but also an exponential decay to zero with a time constant known as T2. the component of magnetization also decays exponentially to the rest magnetization with a time constant known as T1. T2 is known as the spin spin relaxation, and T1 is known as the spin lattice relaxation.

The class of MRI techniques that are sensitive to diffusion are known as DMRI and the setup is as follows, a large coil generates a large constant B field across the entire sample, and the smaller coils are used to generate a gradient such that the protons in the water will only resonate with the chosen radio frequency at a single plane running through the material.

In ordinary MRI one  measures the frequency and the phase of the RF photons released by the atoms and this generates a reciprocal space matrix of which taking the fourier transform generates the real space image of the sample. Now when the protons absorb the radio photons they become initially aligned with the magnetic field which is constant over the plane in which they are excited. this adds a term to our Bloch equation making it the Bloch Torrey equation where the additional term is the divergence of gradient of the magnetization after it has been multiplied by a diffusion tensor, this term represents the change in magnetization due to the random motion of protons through the tissue as atoms that were not excited. Solving the BT equation depends on a couple factors such as the echo time, the gradient field amplitude, the gyrometric factor.

The pulse that excites the protons gives them all the same phase and if you wait an appropriate amount of time the protons in that plane will begin to dephase as they diffuse in and out of that plane. A second radio pulse can then be delivered that would negate the phase of the protons in that plane had they remained stationary a comparison to the signal that would be expected from free water probes the form of the diffusion tensor, or even better we can just allow no time for diffusion in this case we isolate the diffusion tensor, since it is a symmetric matrix we need at least 6 measurements to fully solve for it. This is single shot echo planar imaging.

To discuss models of the diffusion we first define b-value as the sensitivity to diffusion. we find this from the scans potential to provide a large rande of phases to MR signals and is found by integrating the gradient strength over the echo times.
\fi


\if{false}
\subsection{MRI and DMRI}
See my old PHYS 473 Notes

\subsection{Diffusion Tensor Imaging}
See Poljanka Johnson's proposal as well as my notes on the review paper

\subsection{Biophysical Tissue Models}
See my notes on the Alexander 2010 and the review paper and the ISMRM video.\\


Alexander 2010 is:
\emph{Orientationally invariant indices of axon di-
ameter and density from diffusion MRI}

In particular the story we are trying to help along is the idea that once we know how to model diffusion in heterogeneous tissue we then need to make the model we are trying to fit something that is clinically practical, in particular the idea is that we need to not have a strong orientation dependence of the patient and we need to minimize the required data to be taken.
\fi

% MRI gives us the proton density
% DMRI gives us the directions in which water can diffuse
% A biophysical model predicts the structure from the diffusion
% What kind of diffusion we may expect from a region of damaged tissue